\section{S}

    \subsection{Stakeholder}
    \label{glossario:stakeholder}
    Tutti coloro che sono interessati al progetto e che possono influenzarlo:
    \begin{itemize}
        \item \textbf{comunità di utenti} che usa il prodotto;
        \item il \textbf{committente} che compra il prodotto;
        \item il \textbf{fornitore} che sostiene i costi di realizzazione;
        \item eventuali \textbf{regolatori} che verificano la qualità del prodotto;
    \end{itemize}

    \subsection{Software Engineering}
    \label{glossario:softwareengineering}
    Disciplina per la realizzazione di \textbf{prodotti SW} così impegnativi da richiedere il dispiego di attività collaborative.
    Capacità di produrre "in grande" e "in piccolo", garantendo la qualità (\textbf{efficacia}), contenendo il consumo di risorse (\textbf{efficienza}) per l'intero periodo di sviluppo e uso del prodotto (\textbf{ciclo di vita}).

    \subsection{Sistema Qualità}
    \label{glossario:sistemaqualita}
    Struttura organizzativa, responsabilità, procedure e risorse atte al perseguimento della qualità.  
    
\pagebreak