\section{M}

    \subsection{Modelli di ciclo di vita}
    \label{glossario:modellidiciclodivita}
    \begin{itemize}
        \item Descrivono quali stati e transizioni privilegiare in un \hyperref[glossario:ciclodivita]{ciclo di vita} di un prodotto SW.
        \item Aderire a un modello di \hyperref[glossario:ciclodivita]{ciclo di vita} consente di determinare quali processi serva attuare.
        \item E quindi pianificare, organizzare, eseguire e controllare lo svolgimento delle corrispondenti attività.
    \end{itemize}

    \subsection{Misurazione quantitativa}
    \label{glossario:misurazionequantitativa}
    Il processo con cui assegnare simboli o numeri ad attributi di una entità, secondo regole definite.

    \subsection{Milestone}
    \label{glossario:milestone}
    Una data del calendario di progetto, che denota un punto di avanzamento atteso, sostanziato da una baseline corrispondente.

    \subsection{Modulo}
    \label{glossario:modulo}
    È una frazione dell'\hyperref[glossario:unita]{unità}.

\pagebreak