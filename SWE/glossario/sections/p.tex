\section{P}

    \subsection{Progetto}
    \label{glossario:progetto}
    Insieme di attività che:
    \begin{itemize}
        \item devono raggiungere determinati obiettivi a partire da determinate specifiche;
        \item hanno una data di inizio e una data di fine fissate;
        \item dispongono di risorse limitate (tempo, denaro, personale, strumenti, ecc.);
        \item cosumano tali risorse.
    \end{itemize}
    L'outuput di un progetto è un prodotto o un servizio, che soddisfa le specifiche del progetto, composito di:
    \begin{itemize}
        \item sorgente;
        \item eseguibile;
        \item documenti.
    \end{itemize}

    \subsection{Processi di ciclo di vita}
    \label{glossario:processiciclodivita}
    Insieme di attività che devono essere svolte per effettuare corrette transizioni di stati nel \hyperref[glossario:ciclodivita]{ciclo di vita} di un prodotto SW.

    \subsection{Processo}
    \label{glossario:processo}
    Insieme di attività \textbf{correlate} e \textbf{coese} che trasformano input (bisogni) in output (prodotti) secondo regole date, consumando risorse nel farlo.

    \subsection{Prototipo}
    \label{glossario:prototipo}
    Versione preliminare di un prodotto SW con lo scopo di usa-e-getta o per incrementi.

    \subsection{Piano di Qualifica}
    \label{glossario:pianodiqualifica}
    Documento che descrive le attività di qualifica (ovvero di \hyperref[glossario:verifica]{verifica} e \hyperref[glossario:validazione]{validazione}) che devono essere svolte per garantire la qualità del prodotto SW.

    \subsection{Piano della Qualità}
    \label{glossario:pianodellaqualita}
    Le attività del \hyperref[glossario:sistemaqualita]{Sistema Qualità} mirate a fissare gli obiettivi di qualità, insieme con i processi e le risorse necessarie per conseguirli.
        
\pagebreak